\documentclass{sumiilab-paper}

\usepackage[dvipdfmx]{graphicx} % 各種形式の画像を簡単にincludeできます
\usepackage{amsmath,amssymb} % 数式
\usepackage{bm}
\usepackage{mathtools} % 数学記号
\usepackage{cite} % 引用
\usepackage{enumitem} % リスト環境
\usepackage{bussproofs} % 証明器

\usepackage{listings, jvlisting} % ソースコード
% \usepackage{listings-rust}
\usepackage{amsthm} % 定理環境

%% =========================================
%% 定理環境の設定
%% =========================================
\newtheoremstyle{mystyle}% name
{}% space above
{}% space below
{\normalfont}% body font
{}% indent amount
{\bfseries}% theorem head font
{ }% punctuation after theorem head
{4pt}% space after theorem head (default: 5pt)
{\thmname{#1}\thmnumber{#2}\thmnote{\hspace{2pt}(#3)}}% theorem head spec

\theoremstyle{mystyle}
\newtheorem{definition}{定義}
\newtheorem{theorem}[definition]{定理}
\newtheorem{corollary}[definition]{系}
\newtheorem{proposition}[definition]{命題}
\newtheorem{lemma}[definition]{補題}
\newtheorem{example}[definition]{例}
\newtheorem{assumption}[definition]{仮定}
\newtheorem{axiom}[definition]{公理}
\renewcommand{\proofname}{\bf{証明}}
\numberwithin{definition}{chapter} % 定義1.1のように表示

%% ソースコードのキャプション名
\renewcommand{\lstlistingname}{ソースコード}
%% ===============================================
%% 論文の表紙に表示される情報
%% ===============================================

% 論文の年度と種類
\paper{2023 年度 修士論文}

% 論文のタイトル
\title{RustへのFractional Ownershipの\\動的検査の導入}

% 学籍番号と著者のお名前
\author{C2IM1034 馬場 風汰}

% 著者の所属
\institute{東北大学 大学院情報科学研究科\\情報基礎科学専攻}% 修士

% 指導教員のお名前
\supervisor{住井 英二郎 教授}% 指導教員

% 論文発表日時
\date{2023年11月30日 17:55--18:35}
% \date{20XX 年1月1日 \quad 23:00--23:30}
% 発表場所
\venue{オンライン}

%% ===============================================
%% ソースコードの設定
%% ===============================================

% プログラミング言語と表示するフォント等の設定
\lstset{
  language={[Objective]Caml},% プログラミング言語
  basicstyle={\ttfamily\small},% ソースコードのテキストのスタイル
  keywordstyle={\bfseries},% 予約語等のキーワードのスタイル
  commentstyle={},% コメントのスタイル
  stringstyle={},% 文字列のスタイル
  frame=trlb,% ソースコードの枠線の設定 (none だと非表示)
  numbers=left,% 行番号の表示 (none だと非表示)
  numberstyle={\footnotesize},% 行番号のスタイル
  xleftmargin=15pt,% 左余白
  xrightmargin=5pt,% 右余白
  keepspaces=true,% 空白を維持する
  mathescape,% $ で囲った部分を数式として表示する ($ がソースコード中で使えなくなるので注意)
  % 手動強調表示の設定
  moredelim=[is][\bfseries]{@*}{*@},
  moredelim=[is][\itshape]{@/}{/@}
}
\lstMakeShortInline[columns=fullflexible]|% 本文中にコードを|foo|の形式で書くことができます

%% ===============================================
%% 論文中で使う記号とかのマクロ定義
%% ===============================================

%% 論文中で繰り返し使う記号は次のように「マクロ」として実装しておくと良い。
%% TeX ソース中で \BOOL と書くと、\texttt{Bool} に置き換えてくれる。
%% フォントを変え忘れたりするリスクが減るし、あとから記号を変更するのも楽になる。

\newcommand{\bkeyword}[1]{\ensuremath{\mathbf{#1}}}
\newcommand{\BOOL}{\bkeyword{Bool}}
\newcommand{\TRUE}{\bkeyword{true}}
\newcommand{\FALSE}{\bkeyword{false}}
\newcommand{\IF}{\bkeyword{if}}
\newcommand{\THEN}{\bkeyword{then}}
\newcommand{\ELSE}{\bkeyword{else}}

\begin{document}
\frontmatter% ここから前文

\maketitle
% 予備審査の時の要旨
%   Rustはownershipという概念に基づいて安全な静的メモリ管理を行うプログラミング言語である。
% Rustでは、各オブジェクトに唯一のownerが静的に定められている。
% オブジェクトのメモリ領域は、ownerのスコープが終了すると開放される。
% ownershipは、変数への代入の際や関数に引数を渡す際に移動する。
% % *Borrowingの説明*また、オブジェクトが束縛されている変数の(Rustに特有の狭い意味での)参照を作成することで、ownerhsipを一時的に借りることができる。
% % *Lifetimeの説明*Rustでは、参照が有効であるプログラムの文面上の静的な範囲はライフタイムと呼ばれる。

% Rustの通常のスレッド生成では、共有オブジェクトのownershipはいずれか一つのスレッドのみに移動する。
% Rustには静的スコープつきスレッドも提供されているが、
% スレッドの合流が静的スコープに従う場合にしか用いることができない。
% また、RustのArcオブジェクトを用いることでオブジェクトの共有は可能であるが、
% メモリリークが起こる可能性がある。

% 以上のような問題に対処するため、本研究では
% fractional ownership [Boyland 2003]を導入する。
% Fractional ownershipとは、0以上かつ1以下の有理数であり、
% 1はオブジェクトが読み書き可能、0より大きく1未満の値は読み取りのみ可能であることを表す。
% オブジェクト生成時にはownershipとして1を与え、エイリアス生成時にはownershipを分割し、
% エイリアスが消滅する際にはownershipを集約する。
% 本研究ではRustの静的なownershipと、fractional ownershipの動的検査を組み合わせることにより、
% 静的スコープつきスレッドより柔軟に、競合状態を防止する手法を提案・実装する
% ここまで
\begin{abstract}

Rustはownershipという概念に基づいて安全な静的メモリ管理を行うプログラミング言語である。
Rustでは、各オブジェクトに唯一のownerが静的に定められている。
オブジェクトのメモリ領域は、ownerのスコープが終了すると解放される。
ownershipは、変数への代入の際や関数に引数を渡す際に移動する。
% *Borrowingの説明*また、オブジェクトが束縛されている変数の(Rustに特有の狭い意味での)参照を作成することで、ownerhsipを一時的に借りることができる。
% *Lifetimeの説明*Rustでは、参照が有効であるプログラムの文面上の静的な範囲はライフタイムと呼ばれる。

% TODO静的なoenershipだと柔軟じゃないよね...rcがある
% TODO中に別のオブジェクトが存在し、そのオブジェクトの参照の数をカウントする
% TODOクローンもできるよ
RustにはArc (Atomically Reference Counted) オブジェクトが存在する。
Arcオブジェクトは動的に参照の数をカウントし、参照の数が0となったら自動でメモリ領域が解放される。
しかしArcオブジェクトを用いた場合、メモリ領域が自動で解放されるため、
% ユーザの意図したタイミングで解放を保障
% 必要以上に長く生きてしまう可能性がある
メモリリークが起こる可能性がある。

以上のような問題に対処するため、本研究では
fractional ownership [Boyland 2003]を導入する。
Fractional ownershipとは、0以上かつ1以下の有理数であり、
1はオブジェクトが読み書き可能、0より大きく1未満の値は読み取りのみ可能であることを表す。
オブジェクト生成時にはownershipとして1を与え、エイリアス生成時にはownershipを分割し、
エイリアスが消滅する際にはownershipを集約する。
% TODOエイリアスと参照(Rcの部分の)を統一

% TODOfractionalはもともと静的!
本研究ではRustの静的なownershipと、fractional ownershipの動的検査を組み合わせることで
より柔軟に、競合状態を防止する手法を提案・実装する。
% TODO競合状態は関係ない
% TODOメモリリークは検出できる(本来解放されているのに解放されていないと)
% TODOユーザの意図通りに解放できているかどうかを動的に検査
具体的には、メモリ領域を手動で解放するオブジェクトの実装を行った。
この手法は、メモリ領域の解放が起きるタイミングがユーザの意図に反しないため
RustのArcオブジェクトとは異なり、メモリリークを防ぐことができる。
% TODO言い過ぎ(防ぐではなく検出してエラー)
また、静的な検査でなく動的な検査であるため、実行前に検査ができない場合も
柔軟に検査を行うことができる。
% TODO先に言う!
% 解放のタイミングを指定して、本当に解放できるかは動的に検査

\end{abstract}

\tableofcontents% 目次

\mainmatter% ここから本文

\chapter{序論}

\section{背景}
Rustについての簡単な説明
RustのArcの問題点

\section{目的}
Arcより強いメモリ管理の実現によりメモリリークを防ぎたい
% Rustで扱うことができない並列処理を実行できるようにしたい。そのためにfractional Ownershipによる動的検査の導入

\section{本論文の構成}

% %% 参考文献は \cite{ID} とします(ID は refs.bib 内で文献につけた識別子)
% %% BibTeX の使い方などは各自調べて下さい。
% %% 序論とか結論とか \cite{Pierce:TypeSystems}

\chapter{背景}

\section{Rustのownership}

\subsection{ownership}
Rustはownershipという概念に基づいて安全な静的メモリ管理を行っている。
Rustでは、各オブジェクトに唯一のownerが静的に定められている。
オブジェクトのメモリ領域は、ownerのスコープが終了すると開放される。
コード例は以下に示す。
3行目で、変数sはStringオブジェクトのownerとなる。
4行目で、ownerであるsのスコープが終了するためStringオブジェクトのメモリ領域が解放される。
\begin{lstlisting}
fn main() {
  {
    let s = String::from("Hello"); // sがowner
  } // sが指すStringオブジェクトが解放
}
\end{lstlisting}

\subsection{move}
Rustでは、変数への代入や関数に引数を渡す際にownershipが移動する。これをmoveと呼ぶ。
変数への代入の際には以下のコード例のようにmoveが起きる。
3行目でsからnew\_sにownershipがmoveしているため、4行目ではsを利用することができない。
\begin{lstlisting}
fn main() {
  let s = String::from("Hello"); // sがowner
  let new_s = s; // ownershipがnew_sに移動
  s.push_str("World"); // sはownerでないためコンパイルエラー
} // new_sが指すStringオブジェクトが解放    
\end{lstlisting}
また、関数に引数を渡す際には以下のコード例のようにmoveが起きる。
7行目でsから関数fの引数にownershipがmoveしている。
関数fの処理が終了した後、3行目で引数xの指すStringオブジェクトが解放される。
そのため、8行目の時点でsはownerでなくなっておりコンパイルエラーとなる。
\begin{lstlisting}
fn f(mut x: String) {
  x.push_str("World");
} // xの指すStringオブジェクトが解放

fn main() {
  let mut s = String::from("Hello"); // sがowner
  f(s); // ownershipがfの引数に移動
  s.push_str("Oops!"); // sはownerでないためコンパイルエラー
  // Stringオブジェクトは解放されており危険
}
\end{lstlisting}

\subsection{borrowing}
OwnershipはRustに特有の「参照」を作成することで借りることができる。これをborrowingと呼ぶ。

\subsection{lifetime}

\section{Arcについて}

\section{Rustの並列処理}

\subsection{通常のスレッド}

(move, Arc, Mutex)

\subsection{scoped thread}

\section{問題点}

scoped threadでは不自由な例
Arcでメモリリークが起きる場合

\chapter{分数権限を動的に検査する参照オブジェクトの提案}

\section{インターフェース}
新たな参照オブジェクトでは、

\section{実装}

\section{例}

\chapter{考察}

\chapter{結論と今後の課題}
% % \[
% %% 空白を明示的に開けるときは "\," "\ " "~" "\quad" "\qquad" などを使う。
% %% 空白の幅は "\qquad" > "\quad" > "~" = "\ " > "\," の順で大きい。
% %% "~" と "\ " は空白の代わりに改行を許すかどうかの違い("\ " だと改行される可能性がある)
% % \begin{array}{rcl@{\qquad\qquad}r}
% %   t & \Coloneqq & & \text{terms:} \\
% %   & \mid & x & \text{variables} \\
% %   & \mid & \lambda x.~t & \text{lambda abstraction} \\
% %   & \mid & t_1~t_2 & \text{application} \\
% %   & \mid & \TRUE & \text{true} \\
% %   & \mid & \FALSE & \text{false} \\
% %   & \mid & \IF~t_1~\THEN~t_2~\ELSE~t_3 & \text{if statement}
% % \end{array}
% % \]

% % 他にも、次のように、align 環境を使っても、似たようなものを書くことができる。
% % \begin{align}
% %   t \Coloneqq & \tag*{terms:} \\
% %   {}\mid{} & x \tag*{variables} \\
% %   {}\mid{} & \lambda x.~t \tag*{lambda abstraction} \\
% %   {}\mid{} & t_1~t_2 \tag*{application} \\
% %   {}\mid{} & \TRUE \tag*{true} \\
% %   {}\mid{} & \FALSE \tag*{false} \\
% %   {}\mid{} & \IF~t_1~\THEN~t_2~\ELSE~t_3 \tag*{if statement}
% % \end{align}
% % array 環境を愚直に使う場合と比べて、式が中央揃えになるという点と、
% % ``variables'' とかの説明が右端に来ている点が違う。
% % 説明は tag* マクロで出しており、これはもともと式番号を指定するためのものなので、
% % 若干使い方がおかしい気もするが、まぁ、いいだろう。
% % 自分の好みの方を使うと良いだろう。

% % BNF 全体を左揃えにしたいならば、次のように、flalign 環境を使うと良い。
% % align 環境と違って、\verb|&| を余分に1つ付ける必要がある、ということに注意して欲しい(詳しくはソースコードを見よ)。
% % \begin{flalign}
% %   t \Coloneqq & & \tag*{terms:} \\ % & を余分に1つ付けること!
% %   {}\mid{} & x \tag*{variables} \\
% %   {}\mid{} & \lambda x.~t \tag*{lambda abstraction} \\
% %   {}\mid{} & t_1~t_2 \tag*{application} \\
% %   {}\mid{} & \TRUE \tag*{true} \\
% %   {}\mid{} & \FALSE \tag*{false} \\
% %   {}\mid{} & \IF~t_1~\THEN~t_2~\ELSE~t_3 \tag*{if statement}
% % \end{flalign}

% % \section{導出木の書き方の例}

% % 導出木の書き方も色々あるが、ここでは、bussproofs.sty を使った方法を紹介する。
% % 導出木は、手書きでも書きにくいが、\LaTeX だから書きやすいというわけでもなく、
% % (使うパッケージにも依るが)そこそこの苦労は必要である。
% % bussproofs.sty を除く多くの方法では、frac などをベースに「分数」で導出木を書く。
% % bussproofs.sty はこれらとは全く異なるインタフェースであり、慣れれば比較的解りやすい。
% % bussproofs.sty の動作は、(導出木を要素とする)スタックをイメージすると解りやすい。
% % よく使うマクロは次の通り。
% % \begin{itemize}
% % \item \verb|\AxiomC{...}|:Axiom を push する(導出木では葉に相当)
% % \item \verb|\UnaryInfC{...}|:スタックから部分導出木(仮定)を1つ pop して、
% %   それを新たに作ったノード(結論)の子供にすることで、新たな部分導出木を作成し、push する。
% % \item \verb|\BinaryInfC{...}|:スタックから部分導出木(仮定)を2つ pop して、
% %   \verb|\UnaryInfC| と同様の動作を行う。
% % \item \verb|\TrinaryInfC{...}|:スタックから部分導出木(仮定)を3つ pop して、
% %   \verb|\UnaryInfC| と同様の動作を行う。
% % \end{itemize}

% % 実際の使い方は以下の通り。

% % %% T-Var
% % \begin{prooftree}
% %   \AxiomC{$x:T \in \Gamma$}
% %   \RightLabel{\textsc{T-Var}}
% %   \UnaryInfC{$\Gamma \vdash x : T$}
% % \end{prooftree}
% % %% T-Abs
% % \begin{prooftree}
% %   \AxiomC{$\Gamma, x:T \vdash t : U$}
% %   \RightLabel{\textsc{T-Abs}}
% %   \UnaryInfC{$\Gamma \vdash \lambda x.~t : T \to U$}
% % \end{prooftree}
% % %% T-App
% % \begin{prooftree}
% %   \AxiomC{$\Gamma \vdash t_1 : T \to U$}
% %   \AxiomC{$\Gamma \vdash t_2 : T$}
% %   \RightLabel{\textsc{T-App}}
% %   \BinaryInfC{$\Gamma \vdash t_1~t_2 : U$}
% % \end{prooftree}

% % \begin{prooftree}
% %   \AxiomC{}
% %   \RightLabel{\textsc{T-True}}
% %   \UnaryInfC{$x : \BOOL \to \BOOL \vdash \TRUE : \BOOL$}
% %   \RightLabel{\textsc{T-Abs}}
% %   \UnaryInfC{$\vdash \lambda x.~\TRUE : (\BOOL \to \BOOL) \to \BOOL$}
% %   \AxiomC{$y : \BOOL \in y : \BOOL$}
% %   \RightLabel{\textsc{T-Var}}
% %   \UnaryInfC{$y : \BOOL \vdash y : \BOOL$}
% %   \RightLabel{\textsc{T-Abs}}
% %   \UnaryInfC{$\vdash \lambda y.~y : \BOOL \to \BOOL$}
% %   \RightLabel{\textsc{T-App}}
% %   \BinaryInfC{$\vdash (\lambda x.~\TRUE)~(\lambda y.~y) : \BOOL$}
% % \end{prooftree}

% % \section{定理環境}

% % amsthm.styをカスタマイズした定理環境を使う。

% % \begin{theorem}[定理のタイトル]
% %   定理の内容
% % \end{theorem}

% % \begin{lemma}[補題のタイトル]
% %   補題の内容
% % \end{lemma}

% % \begin{corollary}[系のタイトル]
% %   系の内容
% % \end{corollary}

% % \begin{proposition}[命題のタイトル]
% %   命題の内容
% % \end{proposition}

% % \begin{definition}[定義のタイトル]
% %   定義の内容
% % \end{definition}

% % \begin{example}[例のタイトル]
% %   例の内容
% % \end{example}

% % \begin{assumption}[仮定のタイトル]
% %   仮定の内容
% % \end{assumption}

% % \begin{axiom}[公理のタイトル]
% %   公理の内容
% % \end{axiom}

% % \begin{proof}
% %   証明の内容
% % \end{proof}

% % \subsection{定理環境の使い方の例}

% % \begin{lemma}
% %   \label{lem:interesting-lemma}
% %   論文の中で最重要とは言えないような性質・命題は補題 (lemma) にする。
% %   補題や定理から直ちに導けるような軽い命題は系 (corollary) にする(細かい使い分けは人による)。
% % \end{lemma}

% % \begin{proof}
% %   \lstinline|proof*| のように、アスタリスク付きの環境では、番号が付かない。
% % \end{proof}

% % \begin{theorem}
% %   \label{thm:wonderful-theorem}
% %   提案手法の最も重要な性質や命題は、定理 (theorem) として書く。
% %   読者の心をくすぐる興味深いステートメントを書こう。
% % \end{theorem}

% % \begin{proof}
% %   定理 \ref{thm:wonderful-theorem} の華麗な証明。その美しい証明に、読者の目は釘付けだ!
% %   \begin{enumerate}[leftmargin=0pt,itemindent=*,label=Case \arabic*.]
% %   \item 自明
% %   \item 補題 \ref{lem:interesting-lemma} から直ちに導ける。
% %   \item 言うまでもない。目を瞑れば証明が見えてくる。
% %   \item あんまり自明じゃない
% %     \begin{enumerate}[label=(\roman*)]
% %     \item 自明じゃないと思ったけど、やっぱり自明だった
% %     \item ほらね、こんなに簡単
% %     \end{enumerate}
% %   \end{enumerate}
% % \end{proof}

% % \section{ソースコード}

% % ソースコード\ref{src:listup_nodes}は二分木を深さ優先探索して、ノードを列挙する関数である。
% % \begin{lstlisting}[caption=二分木のノードのリストアップ,label=src:listup_nodes]
% % type 'a bin_tree =
% %   | Leaf of 'a
% %   | Node of 'a bin_tree * 'a bin_tree

% % let rec listup_nodes = function
% %   | Leaf x -> [x]
% %   | Node (r, l) -> (listup_nodes r) @ (listup_nodes l)
% % \end{lstlisting}

% % 余談ではあるが、我々の分野ではこういった参照が飛ばされるような(本文から完全に独立した)ソースコードは図として扱い、
% % キャプションを下につける流儀が一般的かと思う。


% % \section{図}

% % 図の貼り方ぐらいも例はあった方がよいかと。
% % さすがに PNG 等は |convert| せずに graphicx でそのまま扱う手法を推奨します:
% % 図\ref{f:aaa}は dblp\_bibtex\_crossref です(少なくとも本文で最初に言及された場所よりも後ろに配置するのが原則です)。

% % もし graphicx を持っていない場合は |tlmgr install graphicx| で入るのでは(知らんけど)。
% % 駄目そうならトップの usepackage と本セクションをコメントアウトした本ファイル
% % を origin に |git push| して引き継ぐようにしてください。

% % \begin{figure}[t]
% %   \centering
% %   \includegraphics[width=.98\linewidth]{../docs/dblp_bibtex_crossref.png}
% % \caption{試しに貼り付けられたdblp\_bibtex\_crossref}
% % \label{f:aaa}
% % \end{figure}

% % \chapter{結論}


% % \backmatter% ここから後付
% % \chapter{謝辞}

% % ステキな論文の謝辞

% % %% 参考文献: bibtex
% % \cite{10.1007/3-540-44898-5_4}
% \bibliographystyle{jplain}
% \bibliography{refs}

% % \appendix% ここから付録
% % \chapter{ステキな付録}
% % 適当な付録。
\end{document}
