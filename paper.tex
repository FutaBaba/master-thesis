\documentclass{sumiilab-paper}

\usepackage[dvipdfmx]{graphicx} % 各種形式の画像を簡単にincludeできます
\usepackage{amsmath,amssymb} % 数式
\usepackage{bm}
\usepackage{mathtools} % 数学記号
\usepackage{cite} % 引用
\usepackage{enumitem} % リスト環境
\usepackage{bussproofs} % 証明器

\usepackage{listings, jvlisting} % ソースコード
\usepackage{plistings}
\usepackage{amsthm} % 定理環境
\usepackage{color}
\usepackage{url}
\usepackage{afterpage}

%% =========================================
%% 定理環境の設定
%% =========================================
\newtheoremstyle{mystyle}% name
{}% space above
{}% space below
{\normalfont}% body font
{}% indent amount
{\bfseries}% theorem head font
{ }% punctuation after theorem head
{4pt}% space after theorem head (default: 5pt)
{\thmname{#1}\thmnumber{#2}\thmnote{\hspace{2pt}(#3)}}% theorem head spec

\theoremstyle{mystyle}
\newtheorem{definition}{定義}
\newtheorem{theorem}[definition]{定理}
\newtheorem{corollary}[definition]{系}
\newtheorem{proposition}[definition]{命題}
\newtheorem{lemma}[definition]{補題}
\newtheorem{example}[definition]{例}
\newtheorem{assumption}[definition]{仮定}
\newtheorem{axiom}[definition]{公理}
\renewcommand{\proofname}{\bf{証明}}
\numberwithin{definition}{chapter} % 定義1.1のように表示

%% ソースコードのキャプション名
\renewcommand{\lstlistingname}{図}
%% ===============================================
%% 論文の表紙に表示される情報
%% ===============================================

% 論文の年度と種類
\paper{2023 年度 修士論文}

% 論文のタイトル
\title{RustへのFractional Ownershipの\\動的検査の導入}

% 学籍番号と著者のお名前
\author{C2IM1034 馬場 風汰}

% 著者の所属
\institute{東北大学 大学院情報科学研究科\\情報基礎科学専攻}% 修士

% 指導教員のお名前
\supervisor{住井 英二郎 教授}% 指導教員

% 論文発表日時
\date{2024年2月7日 10:50--11:50}
% \date{20XX 年1月1日 \quad 23:00--23:30}
% 発表場所
\venue{オンライン}

%% ===============================================
%% ソースコードの設定
%% ===============================================

% プログラミング言語と表示するフォント等の設定
\lstset{
  language={},% プログラミング言語
  basicstyle={\ttfamily\small},% ソースコードのテキストのスタイル
  keywordstyle={\bfseries},% 予約語等のキーワードのスタイル
  commentstyle={\ttfamily\small},% コメントのスタイル
  morecomment = [l]{//},
  stringstyle={},% 文字列のスタイル
  frame=trlb,% ソースコードの枠線の設定 (none だと非表示)
  numbers=left,% 行番号の表示 (none だと非表示)
  numberstyle={\footnotesize},% 行番号のスタイル
  xleftmargin=15pt,% 左余白
  xrightmargin=5pt,% 右余白
  keepspaces=true,% 空白を維持する
  % mathescape,% $ で囲った部分を数式として表示する ($ がソースコード中で使えなくなるので注意)
  % 手動強調表示の設定
  moredelim=[is][\color{red}\bfseries]{@*}{*@},
  moredelim=[is][\itshape]{@/}{/@}
}
\lstMakeShortInline[columns=fullflexible]|% 本文中にコードを|foo|の形式で書くことができます

%% ===============================================
%% 論文中で使う記号とかのマクロ定義
%% ===============================================

%% 論文中で繰り返し使う記号は次のように「マクロ」として実装しておくと良い。
%% TeX ソース中で \BOOL と書くと、\texttt{Bool} に置き換えてくれる。
%% フォントを変え忘れたりするリスクが減るし、あとから記号を変更するのも楽になる。

\newcommand{\bkeyword}[1]{\ensuremath{\mathbf{#1}}}
\newcommand{\BOOL}{\bkeyword{Bool}}
\newcommand{\TRUE}{\bkeyword{true}}
\newcommand{\FALSE}{\bkeyword{false}}
\newcommand{\IF}{\bkeyword{if}}
\newcommand{\THEN}{\bkeyword{then}}
\newcommand{\ELSE}{\bkeyword{else}}

\begin{document}
\frontmatter% ここから前文
% \maketitle
% 予備審査の時の要旨
%   Rustはownershipという概念に基づいて安全な静的メモリ管理を行うプログラミング言語である。
% Rustでは、各オブジェクトに唯一のownerが静的に定められている。
% オブジェクトのメモリ領域は、ownerのスコープが終了すると開放される。
% ownershipは、変数への代入の際や関数に引数を渡す際に移動する。
% % *Borrowingの説明*また、オブジェクトが束縛されている変数の(Rustに特有の狭い意味での)参照を作成することで、ownerhsipを一時的に借りることができる。
% % *Lifetimeの説明*Rustでは、参照が有効であるプログラムの文面上の静的な範囲はライフタイムと呼ばれる。

% Rustの通常のスレッド生成では、共有オブジェクトのownershipはいずれか一つのスレッドのみに移動する。
% Rustには静的スコープつきスレッドも提供されているが、
% スレッドの合流が静的スコープに従う場合にしか用いることができない。
% また、RustのArcオブジェクトを用いることでオブジェクトの共有は可能であるが、
% メモリリークが起こる可能性がある。

% 以上のような問題に対処するため、本研究では
% fractional ownership [Boyland 2003]を導入する。
% Fractional ownershipとは、0以上かつ1以下の有理数であり、
% 1はオブジェクトが読み書き可能、0より大きく1未満の値は読み取りのみ可能であることを表す。
% オブジェクト生成時にはownershipとして1を与え、エイリアス生成時にはownershipを分割し、
% エイリアスが消滅する際にはownershipを集約する。
% 本研究ではRustの静的なownershipと、fractional ownershipの動的検査を組み合わせることにより、
% 静的スコープつきスレッドより柔軟に、競合状態を防止する手法を提案・実装する
% ここまで
\begin{abstract}
% 0124 check
Rustはownership (所有権)の概念に基づいて安全な静的メモリ管理を行うプログラミング言語である。
Rustでは、各オブジェクトに唯一のownerが静的に定められている。
Ownershipは、変数への代入や関数に引数を渡す際に移動する。
オブジェクトのメモリ領域は、ownerのスコープが終了すると解放される。
% *Borrowingの説明*また、オブジェクトが束縛されている変数の(Rustに特有の狭い意味での)参照を作成することで、ownerhsipを一時的に借りることができる。
% *Lifetimeの説明*Rustでは、参照が有効であるプログラムの文面上の静的な範囲はライフタイムと呼ばれる。

しかし、静的なownershipのみによるメモリ管理は柔軟性に欠ける場合がある。
そのためRustには
\texttt{Rc} (参照カウント) オブジェクトも存在する。
\texttt{Rc}オブジェクトはコンテナであり、中身のオブジェクトへの参照を持つ。
\texttt{Rc}オブジェクトはclone関数により複製でき、その際共通の参照カウントを1増やす。
複製が消滅する際には参照カウントを1減らし、
参照カウントが0になったら中身のオブジェクトを解放する。
そのため、
意図せず複製が残っていると
メモリリークのおそれがある。

そこで
本研究ではfractional ownership [Boyland 2003]の考え方を取り入れて、
Rustの静的なownershipによるメモリ管理と、動的な参照カウントの柔軟性を組み合わせる。
Fractional ownershipとは、0以上かつ1以下の有理数である。
新たに生成されるオブジェクトにはownershipとして1が与えられる。
参照複製時にはownershipが分割される。
複製された参照が消滅する際にはownershipは集約される。
ownershipが0より大きいオブジェクトは、
いずれ解放される必要があるが、
オブジェクトの解放にはownership 1が必要である。
本来fractional ownershipは静的に検査されるが、本研究では参照カウントを
動的なfractional ownershipとみなす方式を提案・実装する。
これにより、動的な参照カウントの柔軟性を活用した後、
参照カウントが1になったオブジェクトはfractional ownership 1とみなし、
Rustの静的なownershipに戻すことができる。
逆に、静的なownershipに戻さないまま参照が消滅したら実行時エラーとすることにより、
オブジェクトの解放し忘れを検出することができる。
% 厳密な世界と動的な世界を行き来 dropの際は厳密な世界に戻らないといけない(新しい点ではない)

\end{abstract}

\tableofcontents% 目次

\mainmatter% ここから本文

\chapter{序論}
\section{背景}
Rust\cite{DBLP:conf/sigada/MatsakisK14, rust-web, rust-book}は、
ownership\cite{DBLP:conf/oopsla/ClarkePN98}の概念に基づき
% TODO ownershipの他の主な文献
静的なメモリ管理を行うプログラミング言語であり、
Redox OS\cite{redox}や、
Dropbox\cite{dropbox}、
Firefox\cite{fire}
といったシステムプログラミング等に用いられている\cite{rust-web}。
% もっとリファレンス入れてみる
Rust登場以前は、メモリ管理はユーザが手動で管理する方式
\cite{c-book}
と実行時ガベージコレクタ\cite{gc-book}
を用いた方式の2つが
主なものであった。
しかしRustのメモリ管理は、手動のメモリ管理と異なり、解放し忘れによるメモリリークなどをコンパイル時に検出でき、
さらに、実行時ガベージコレクタを用いたメモリ管理より効率が良い。
% TODO 効率がよいことのリファレンス
そのため、ユーザはメモリ管理という低レベルの制御を、これまでの方式と比べて楽に効率よく行うことができる。

しかし、Rustのメモリ管理は静的であり、動的なメモリ管理と比べると柔軟性に欠ける場合がある。
そのためRustには、静的なownershipを無視する抜け道としてunsafeな機能が存在する。
unsafeな機能が内部の実装で用いられているオブジェクトの一つに
\texttt{Rc}オブジェクト\cite{rc}が存在する。
\texttt{Rc}オブジェクトは、中身のオブジェクトへの参照と、共通の参照カウントを持つコンテナであり、動的な参照カウントにより静的なownershipを補っている。
clone関数により複製が可能であり、その際共通の参照カウントを1増やし、参照が消滅する際には参照カウントを1減らし、参照カウントが0になったときに中身のオブジェクトを解放する。
本来Rustでは、ownershipの複製をすることはできないが、\texttt{Rc}オブジェクトの内部の実装で用いられているunsafeな機能によりownershipの複製ができるようになる。
なお、Rustの静的なownershipを無視するunsafeな機能が用いられているので、\texttt{Rc}オブジェクトは動的な柔軟性が必要な場合にのみ用いられるべきである。

図\ref{rc_simple}は、
静的なownershipのみではメモリ管理が難しい例であり、
\texttt{Rc}オブジェクトの動的な参照カウントの柔軟性を活用した例である。
中身の値が\texttt{"Hello"}という文字列である
\texttt{String}オブジェクトを持つ
\texttt{Rc}オブジェクト\texttt{h}と、
中身の値が\texttt{"World"}という文字列である
\texttt{String}オブジェクトを持つ
\texttt{Rc}オブジェクト\texttt{w}の複製をランダムに10回配列に入れ、
その配列から中身の値が\texttt{"Hello"}という文字列である\texttt{String}オブジェクトを持つ\texttt{Rc}オブジェクトを
削除するプログラムである。
このプログラムは、
オブジェクトの生存区間が中身の値に依存するため、静的なメモリ管理は難しい。
\begin{figure}[htp]
\begin{lstlisting}[caption=静的なメモリ管理では柔軟性に欠ける例, label=rc_simple, captionpos=b]
fn main() {
    let mut vec: Vec<Rc<String>> = Vec::new();
    let h = Rc::new(String::from("Hello"));
    let w = Rc::new(String::from("World"));
    let mut rng = rand::thread_rng();
    for _ in 0..10 {
        if rng.gen() {
            vec.push(h.clone());
        }
        else {
            vec.push(w.clone());
        }
    }
    remove_string(&mut vec, String::from("Hello"));
    // その他の処理
}
\end{lstlisting}
\end{figure}

しかし、図\ref{rc_simple}の17行目において、
その他の処理として、\texttt{vec}は用いるが、\texttt{h}と\texttt{w}
は用いない処理が行われる場合、
\texttt{h}のメモリ領域は、追加した処理の最中も確保されたままと
なってしまう。
このように、\texttt{Rc}オブジェクトを用いた場合、
意図せず複製が残ってしまうことにより、
メモリリークのおそれがあり、
Rustの、メモリリークを防止できるというメリットが損なわれてしまう。
このメモリリークは、後ほど第3章で述べるように、
3行目から16行目を別のスコープにすることで防ぐことができる。
また、同様に第3章で詳しく述べているが、
\texttt{into\_inner}関数を用いることでこのようなメモリリークを検出することもできる。
しかし、ユーザのミスにより3行目から16行目を別のスコープに
し忘れた場合や、\texttt{into\_inner}関数を用いることを忘れた場合には、メモリリークが起きているにも関わらず、
静的にも動的にも検出することはできない。

また、\texttt{into\_inner}関数の使用忘れやスコープの切り忘れを防止するために、それらをまとめた高階関数を用意することが考えられるが、
Rustのように静的なownershipに基づきメモリ管理を行う言語では、高階関数に引数として渡す関数内で用いられているオブジェクトのborrowingのlifetimeの推論や指定方法は自明でないという問題がある。

\section{本研究の概略}
本研究では、柔軟な複製を許すものの、ユーザのミスによりメモリリークが起きやすい\texttt{Rc}オブジェクトに代わり、
柔軟なエイリアスを許しつつ、メモリリークの検出が可能な仕組みを実現することを目的とする。
具体的には、
fractional ownership\cite{DBLP:conf/sas/Boyland03}の考え方を取り入れ、
Rustの静的なownershipによるメモリ管理と、
動的な参照カウントの柔軟性を組み合わせた新たな参照オブジェクトを提案する。
提案する参照オブジェクトの仕組みにより、どのオブジェクトがどこで解放されるはずかが、
スコープが終了するとき、もしくは\texttt{drop}関数が明示的に呼び出されるときのいずれかとなるため、
解放のタイミングがプログラムの文面上明確になる。
また、意図せず複製が残っていることで中身のオブジェクトが解放されない可能性がある\texttt{Rc}オブジェクトと異なり、
提案する参照オブジェクトは実際に解放されなければ実行時エラーが起きるため、ユーザのミスによるオブジェクトの解放し忘れを検出できる。

Fractional ownershipとは、0以上かつ1以下の有理数である。
新たに生成されるオブジェクトにはownershipとして1が与えられる。
参照複製時にはownershipが分割される。
複製された参照が消滅する際にはownershipは集約される。
ownershipが0より大きいオブジェクトは、
いずれ解放される必要があるが、
オブジェクトの解放にはownership 1が必要である。
本来fractional ownership\cite{DBLP:conf/sas/Boyland03}は静的に検査されるが、
我々の参照オブジェクトは、参照カウントの逆数を動的なfractional ownershipとみなしている。
例えば、参照カウントが2であるとき、fractional ownershipは1/2とみなす。

図\ref{random_simple}は、図\ref{rc_simple}のプログラムを
本研究の参照オブジェクトで表したプログラムである。
本研究の参照オブジェクトは、本来のRustのownershipに従うmutableな参照オブジェクトと、Rustのownershipに違反して複製が可能なimmutableな参照オブジェクトに区別されている。
動的な参照カウントによる柔軟性を活用する際には、
6行目、7行目の\texttt{to\_immut}関数を用いてmutableな参照からimmutableな参照に変換し、
活用後、参照カウントが1であれば
20行目、23行目の\texttt{back\_to\_mut}関数を用いてmutableな参照に戻すことができる。
\begin{figure}[htp]
\begin{lstlisting}[caption=本研究の方式を用いた具体例, 
  label=random_simple, captionpos=b]
fn main() {
    let mut vec: Vec<FRefImmut<String>> = Vec::new();
    let mut h = FRefMut::new(String::from("Hello"));
    let mut w = FRefMut::new(String::from"World"));
    let mut rng = rand::thread_rng();
    let h_immut = h.to_immut();
    let w_immut = w.to_immut();
    for _ in 0..10 {
        if rng.gen() {
            let h_immut1 = h_immut.clone_immut();
            vec.push(h_immut1);
        }
        else {
            let w_immut1 = w_immut.clone_immut();
            vec.push(w_immut1);
        }
    }

    remove_string(&mut vec, String::from("Hello"));
    let mut h_mut = h_immut.back_to_mut();
    drop(h_mut);
    drop(vec);
    let mut w_mut = w_immut.back_to_mut();
}
\end{lstlisting}
\end{figure}

またこの参照オブジェクトは、図\ref{error_simple}のようにmutableな参照に戻さないまま消滅した場合、実行時エラーが起きる。
これにより、意図せず参照の複製が残っていることによるオブジェクトの解放し忘れを
実行時に検出することができる。

\begin{figure}[htp]
\begin{lstlisting}[caption=メモリリークを検出する例, 
  label=error_simple, captionpos=b]
fn main() {
    let mut vec: Vec<FRefImmut<String>> = Vec::new();
    let mut h = FRefMut::new(String::from("Hello"));
    let mut w = FRefMut::new(String::from("World"));
    let mut rng = rand::thread_rng();
    let h_immut = h.to_immut();
    let w_immut = w.to_immut();
    for _ in 0..10 {
        if rng.gen() {
            let h_immut1 = h_immut.clone_immut();
            vec.push(h_immut1);
        }
        else {
            let w_immut1 = w_immut.clone_immut();
            vec.push(w_immut1);
        }
    }

    remove_string(&mut vec, String::from("Hello"));
    // Mutableに戻し忘れ!
}
\end{lstlisting}
\end{figure}

このように我々の参照オブジェクトは、
mutableな参照の時に中身のオブジェクトのメモリ領域が解放されていることを動的に検査している。
この参照オブジェクトは、必要な時に動的な参照カウントの柔軟性を利用することが
できるため、静的なownershipのみでメモリ管理行うことが難しい場合に用いることも可能である。本研究では、動的な参照カウントの柔軟性が必要な場合として、特に並列処理に我々の参照オブジェクトを用いた場合について述べている。詳しくは、第5章で述べる。

\section{本論文の構成}
本論文では、まず第2章でRustのownership、\texttt{Rc}オブジェクト、fractional ownershipについて
述べた後、第3章で\texttt{Rc}オブジェクトを用いる際の問題点について述べた後、
第4章で本研究で提案する新たな参照オブジェクトについて述べる。
その後第5章で新たな参照オブジェクトの並列処理への応用について述べ、
第6章で関連研究、第7章で結論と今後の課題を述べる。

% %% 参考文献は \cite{ID} とします(ID は refs.bib 内で文献につけた識別子)
% %% BibTeX の使い方などは各自調べて下さい。

\chapter{背景}

\section{Rustのownership}

\subsection{Ownership}
% 1/20 check
% TODO RustBeltでownerはどうなってるか確認 ownerはアクセスが許されている参照
Rustはownershipという概念に基づいて安全な静的メモリ管理を行っている。
Rustでは、各オブジェクトに唯一のownerが静的に定められており、
オブジェクトのメモリ領域は、ownerのスコープが終了すると開放される。

実際の例を
図\ref{ownership_nest_example}に示す。
3行目で変数\texttt{s}は\texttt{"Hello"}という文字列を保持する\texttt{String}オブジェクトのownerとなり、
5行目で変数\texttt{t}は\texttt{"World"}という文字列を保持する\texttt{String}オブジェクトのownerとなる。
その後6行目で、
\texttt{t}のスコープが終了するため、\texttt{"World"}という文字列を保持する\texttt{String}オブジェクトの
メモリ領域が先に解放された後、
7行目で、\texttt{s}のスコープが終了するため、\texttt{"World"}という文字列を保持する\texttt{String}オブジェクトの
メモリ領域が解放される。
\begin{figure}[htp]
\begin{lstlisting}[caption=Ownershipによるメモリ管理,label=ownership_nest_example, captionpos=b]
fn main() {
  {
    let s = String::from("Hello"); // sがStringオブジェクトのowner
    {
      let t = String::from("World"); // tがStringオブジェクトのowner
    } // tが指すStringオブジェクトが解放
  } // sが指すStringオブジェクトが解放
}
\end{lstlisting}
\end{figure}

\subsection{Move}
% 0120 check
Rustでは、変数への代入や関数に引数を渡す際にownershipが移動する。これをmoveと呼ぶ。
変数への代入の際には図\ref{move_assign_example}のようにmoveが起きる。
2行目で、変数\texttt{s}が\texttt{String}オブジェクトのownerとなり、
3行目で\texttt{s}から\texttt{new\_s}にownershipがmoveしている。
Ownershipがmoveしたことで\texttt{s}はownerでなくなるため、
4行目で\texttt{s}を利用しようとするとコンパイルエラーが発生する。
\begin{figure}[htp]
\begin{lstlisting}[caption=変数への代入でmoveが起こる例,
  label=move_assign_example, captionpos=b]
fn main() {
  let mut s = String::from("Hello"); // sがStringオブジェクトのowner
  let new_s = s; // sからnew_sにownershipがmove
  s.push_str("World"); // コンパイルエラー
}    
\end{lstlisting}
\end{figure}

また、関数に引数を渡す際には以下の図\ref{move_function_example}のようにmoveが起きる。
6行目で、変数\texttt{s}が\texttt{String}オブジェクトのownerとなり、
7行目で\texttt{s}から関数\texttt{f}の引数xにownershipがmoveしている。
Ownershipがmoveしているため、\texttt{x}は\texttt{String}オブジェクトのownerとなり、
2行目の処理を行うことができる。
関数\texttt{f}の処理が終了した後、3行目で引数\texttt{x}の指す\texttt{String}オブジェクトが解放される。
そのため、8行目の時点で\texttt{s}はownerでないためコンパイルエラーが発生する。
さらに、8行目ですでに解放されている\texttt{String}オブジェクトに対して処理を行おうとしているため、
実際に危険な処理となっている。
\begin{figure}[htp]
\begin{lstlisting}[caption=関数に引数を渡す際にmoveが起こる例, 
label=move_function_example, captionpos=b]
fn f(mut x: String) {
  x.push_str("World"); // xはStringオブジェクトのowner
} // xの指すStringオブジェクトが解放

fn main() {
  let mut s = String::from("Hello"); // sがStringオブジェクトのowner
  f(s); // sから関数fの引数xにownershipがmove
  s.push_str("Oops!"); // コンパイルエラー
}
\end{lstlisting}
\end{figure}

図\ref{move_function_example}のプログラムは、図\ref{move_return_function}のように
3行目で関数fが引数xをownershipごと返却するように変更することで、
コンパイルエラーを発生させずに実行することができる。
しかし、関数の引数にownershipがmoveするたびに、
ownershipを返却するようにプログラムを書くことは面倒である。
\afterpage{\clearpage}
\newpage

\begin{figure}[htp]
\begin{lstlisting}[caption=Ownershipを返却する例, 
  label=move_return_function, captionpos=b]
fn f(mut x: String) @*-> String*@{
  x.push_str("World"); // xはStringオブジェクトのowner
  @*return x*@ // ownershipを返却
}
  
fn main() {
  let mut s = String::from("Hello"); // sがStringオブジェクトのowner
  @*let mut return_s = *@f(s); // sから関数fの引数xにownershipがmove
  return_s.push_str("World");
}
\end{lstlisting}
\end{figure}

\subsection{Borrowing}
% 0120 check
Moveの際の不便な点を解消するため、
Rustにはownershipを一時的に借り、自動で返却する仕組みがある。
Ownershipを借りる行為はborrowingと呼ばれ、
Rustに特有の「参照」を作成することで借りることができる。
変数名の前に\texttt{\&}とつけることで、
Rustに特有の「参照」を作成し、ownershipをborrowingできる。

Ownershipをborrowingしている例を図\ref{borrowing_example}に示す。
6行目で、変数\texttt{s}が\texttt{String}オブジェクトのownerとなり、
7行目で\texttt{s}のownershipが一時的に関数\texttt{f}の引数\texttt{x}に貸しだされる。
Ownershipが貸しだされているため、\texttt{x}は\texttt{String}オブジェクトのownerとなり、
2行目の処理を行うことができる。
関数\texttt{f}の処理が終了した後、\texttt{x}に貸しだされていたownershipは
自動で変数\texttt{s}に返却される。
そのため、8行目で\texttt{s}はownerであり、\texttt{s}に対して処理が可能である。

\begin{figure}[htp]
\begin{lstlisting}[caption=Borrowingの例, label=borrowing_example, captionpos=b]
fn f(x: &String) {
  println!("{}", &x); // xはStringオブジェクトのowner
} // sにownershipが自動で返却

fn main() {
  let mut s = String::from("Hello"); // sがStringオブジェクトのowner
  f(&s); // sのownershipが一時的に関数fの引数xに貸し出される
  s.push_str("World"); // sはStringオブジェクトのowner
}
\end{lstlisting}
\end{figure}

\subsection{Mutableなborrowing}
% 0120 check
上で述べたborrowingは、読み取りのみ可能であるimmutableなborrowing
であり、読み書きが可能であるmutableなborrowingも存在する。
変数名の前に\texttt{\&mut}とつけるとmutableなborroiwngができる。
Mutableなborrowingをしている例を図\ref{mutable_borrowing_example}に示す。
6行目で、変数\texttt{s}が\texttt{String}オブジェクトのownerとなり、
7行目で\texttt{s}のownershipが一時的に関数\texttt{f}の引数\texttt{x}に貸しだされる。
Ownershipが貸しだされているため、\texttt{x}は\texttt{String}オブジェクトのownerとなり、
2行目の処理を行うことができる。
関数\texttt{f}の処理が終了した後、\texttt{x}に貸しだされていたownershipは
自動で変数\texttt{s}に返却される。
そのため、8行目で\texttt{s}はownerであり、\texttt{s}に対して処理が可能である。

\begin{figure}[htp]
\begin{lstlisting}[caption=Mutableなborrowingの例, label=mutable_borrowing_example, captionpos=b]
fn f(x: &@*mut*@ String) {
  x.push_str("hoge") // xはStringオブジェクトのowner
} // sにownershipが自動で返却

fn main() {
  let mut s = String::from("Hello"); // sがStringオブジェクトのowner
  f(&@*mut*@ s); // sのownershipが一時的に関数fの引数xに貸し出される
  s.push_str("World"); // sはStringオブジェクトのowner
}
\end{lstlisting}
\end{figure}

Immutableなborrowingは、すでにmutableなborrowingが行われていなければ
複数回行うことができる。
逆にmutableなborrowingは、mutableなborrowingも
immutableなborrowingも行われていないときにしか行うことはできない。

\subsection{lifetime}
% 0120 check
Borrowingによって貸し出されたownershipは、lifetimeという仕組みを利用して自動で返却される。
Lifetimeとは、参照を用いることができる範囲のことであり、基本的に自動で推論される。
Lifetimeは、図\ref{lifetime_example}のように、明示的に注釈をつけることもできる。
1行目のように書くことで、引数xのlifetimeが関数fの本体と一致していると注釈でき、
関数fの処理が終了したとき、xのlifetimeも終了するため、その時点でownershipが返却される。
\begin{figure}[htp]
\begin{lstlisting}[caption=Lifetime注釈, label=lifetime_example, captionpos=b]
fn f<'a>(x: &'a String) { // xのライフタイムと関数本体が一致
  println!("{}", &x);
} // sにownershipが自動で返却

fn main() {
  let mut s = String::from("Hello"); // sがStringオブジェクトのowner
  f(&s); // sのownershipが一時的に関数fの引数xに貸し出される
  s.push_str("World"); // sはStringオブジェクトのowner
}
\end{lstlisting}
\end{figure}

図\ref{lifetime_example2}は、スコープとlifetimeが一致していない例である。
このとき、7行目以降では\texttt{s\_borrow}は用いられていないため、\texttt{s\_borrow}のライフタイムは4行目から6行目と推論され、\texttt{t\_borrow}のライフタイムは7行目から8行目で\texttt{main}関数のスコープが終了するまでであると推論される。
このように、ライフタイムはブロックにより推論されているわけではなく、
スコープとライフタイムが一致しない場合もある。
この推論は、RustのNon-lexical lifetimes (NLL)\cite{nll}により可能となっている。
\begin{figure}[htp]
\begin{lstlisting}[caption=スコープとlifetimeが一致していない例, label=lifetime_example2, captionpos=b]
fn main () {
    let mut s = String::from("Hello");

    let s_borrow = &mut s;
    println!("{}", s_borrow);
    // s_borrowのライフタイム終了
    let t_borrow = &mut s; // Mutableなborrowingが可能
} // t_borrowのライフタイム終了
\end{lstlisting}
\end{figure}

\section{Rustの参照カウント}
% 0120 check
後述の例のように、
Rustの静的なownershipのみによるメモリ管理は柔軟性に欠ける場合があるため、
Rustには\texttt{Rc} (参照カウント) オブジェクトも存在する。
\texttt{Rc}オブジェクトはコンテナであり、中身のオブジェクトへの参照と、共通の参照カウントを持つ。
また、\texttt{clone}関数により複製でき、その際共通の参照カウントを1増やす。
複製が消滅する際には参照カウントを1減らし、
参照カウントが0になったら中身のオブジェクトを解放する。
\texttt{Rc}オブジェクトの実装には、Rustの静的なownershipを無視するunsafeな機能が用いられている。

\texttt{Rc}オブジェクトは、図\ref{rc_example}のように用いられる。
このプログラムは、
中身に\texttt{"Hello"}という\texttt{String}オブジェクトを持つ
\texttt{Rc}オブジェクト\texttt{h}と、中身に\texttt{"World"}という\texttt{String}オブジェクトを持つ
\texttt{Rc}オブジェクト\texttt{w}の複製をランダムに10回配列に入れ、
その配列から\texttt{"Hello"}という\texttt{String}オブジェクトを持つ\texttt{Rc}オブジェクトを
削除するプログラムである。
オブジェクトの生存期間が中身の値に依存しているため、静的なownershipのみではメモリ管理が難しい例である。
12行目と16行目では\texttt{clone}関数により、\texttt{Rc}オブジェクトの複製を生成し、参照カウントが1増えている。
20行目で\texttt{h}の複製は1つを残してすべて消滅し、\texttt{h}の参照カウントは1まで減少する。
最後に21行目で\texttt{h}、\texttt{w}の参照カウントは0となり、中身のオブジェクトが解放される。
\begin{figure}[htp]
\begin{lstlisting}[caption=Rcオブジェクトの利用例, label=rc_example, captionpos=b]
use std::rc::Rc;
use rand::Rng;
fn remove_string(vec: &mut Vec<Rc<String>>, s: String) {
    vec.retain(|x| **x != s);
}

fn main() {
    let mut vec: Vec<Rc<String>> = Vec::new();
    let h = Rc::new(String::from("Hello"));
    let w = Rc::new(String::from("World"));
    let mut rng = rand::thread_rng();
    for _ in 0..10 {
        if rng.gen() {
            let h_clone = h.clone(); // clone関数により複製を生成
            vec.push(h_clone);
        }
        else {
            let w_clone = w.clone(); // clone関数により複製を生成
            vec.push(w_clone);
        }
    }
    remove_string(&mut vec, String::from("Hello")); // hの複製が消滅
} // hとwの中身のオブジェクトが解放
\end{lstlisting}
\end{figure}

\afterpage{\clearpage}
\newpage
\section{静的なfractional ownership}
% 0120 check
一方、Rustのownershipとは別にfractional ownership\cite{DBLP:conf/sas/Boyland03}
という概念が存在する。
Fractional ownershipとは0以上かつ1以下の有理数である。
オブジェクト生成時にはownershipとして1が与えられ、参照複製時にはownershipは分割され、
複製された参照が消滅する際にはownershipは集約される。
Ownershipが0より大きいオブジェクトはいずれ解放される必要があるが、
オブジェクトの解放にはownership 1が必要である。
Ownershipが0より大きく1未満のオブジェクトは、
ownershipの集約によりownershipが1となれば解放できる。
図\ref{fractional_example}は、fractional ownershipを利用した疑似コード例である。
1行目で、変数\texttt{a}、\texttt{b}、\texttt{c}に
ownershipとして1を与え、2行目では、dとaでownershipを0.5ずつに
分割している。3行目では\texttt{*b += *a}と\texttt{*c += *d}は
並列に処理されており、\texttt{b}、\texttt{c}はownership 1を持つため読み書き可能であり、
\texttt{a}、\texttt{d}はownership 0.5を持つため読み取りのみ可能になる。
4行目で、分割されていたownershipは\texttt{a}に集約され、
\texttt{a}は読み書き可能となる。
\begin{figure}[htp]
\begin{lstlisting}[caption=Fractional ownershipの利用例, label=fractional_example, captionpos=b]
a := new; b := new; c := new; // Ownership 1を与える
d = a; // dとaでownershipを0.5ずつに分割
*b += *a || *c += *d // b、cは読み書き可能、a、dは読み取りのみ可能
*a += *b + *c // a、b、cは読み書き可能
\end{lstlisting}
\end{figure}

\chapter{問題点}
% 0120 check
図\ref{rc_problem}のプログラムは、
中身に\texttt{"Hello"}という\texttt{String}オブジェクトを持つ
\texttt{Rc}オブジェクト\texttt{h}と、中身に\texttt{"World"}という\texttt{String}オブジェクトを持つ
\texttt{Rc}オブジェクト\texttt{w}の複製をランダムに10回配列に入れ、
その配列から\texttt{"Hello"}という\texttt{String}オブジェクトを持つ\texttt{Rc}オブジェクトを
削除した後、他の処理を行うプログラムである。
23行目にその他の処理として、配列\texttt{vec}は用いるが、
\texttt{Rc}オブジェクト\texttt{h}と\texttt{Rc}オブジェクト\texttt{w}を用いない処理が
行われているとする。
このとき、2つの\texttt{Rc}オブジェクトの中身が解放されるのは24行目で\texttt{main}関数の
スコープが終了したときであり、追加した新たな処理で\texttt{h}
を利用していないにもかかわらず\texttt{h}のメモリ領域が確保されたままとなってしまう。
このように、\texttt{Rc}オブジェクトを用いた場合、意図せずメモリリークが発生するおそれがある。
\begin{figure}[htp]
\begin{lstlisting}[caption=Rcオブジェクトを利用する際の問題点, label=rc_problem, captionpos=b]
use std::rc::Rc;
use rand::Rng;
fn remove_string(vec: &mut Vec<Rc<String>>, s: String) {
    vec.retain(|x| **x != s);
}
  
fn main() {
    let mut vec: Vec<Rc<String>> = Vec::new();
    let h = Rc::new(String::from("Hello"));
    let w = Rc::new(String::from("World"));
    let mut rng = rand::thread_rng();
    for _ in 0..10 {
        if rng.gen() {
            let h_clone = h.clone();
            vec.push(h_clone);
        }
        else {
            let w_clone = w.clone();
            vec.push(w_clone);
        }
    }
    remove_string(&mut vec, String::from("Hello"));
    // その他の処理
} // hとwの中身のオブジェクトが解放
\end{lstlisting}
\end{figure}

図\ref{rc_scope_problem}のプログラムは、
24行目で\texttt{h}と\texttt{w}のスコープが終了し、\texttt{h}のメモリ領域が解放されるため、
図\ref{rc_problem}のような意図していないメモリリークを
防いだプログラムである。
\begin{figure}[htp]
\begin{lstlisting}[caption=メモリリークを防いだ例, 
  label=rc_scope_problem, captionpos=b]
use std::rc::Rc;
use rand::Rng;
fn remove_string(vec: &mut Vec<Rc<String>>, s: String) {
    vec.retain(|x| **x != s);
}
  
fn main() {
    let mut vec: Vec<Rc<String>> = Vec::new();
    @*{*@
      let h = Rc::new(String::from("Hello"));
      let w = Rc::new(String::from("World"));
      let mut rng = rand::thread_rng();
      for _ in 0..10 {
          if rng.gen() {
              let h_clone = h.clone();
              vec.push(h_clone);
          }
          else {
              let w_clone = w.clone();
              vec.push(w_clone);
          }
      } 
      remove_string(&mut vec, String::from("Hello"));
    @*}*@ // hの中身のオブジェクトが解放
    // その他の処理
}
\end{lstlisting}
\end{figure}

しかし、例えば図\ref{rc_miss_problem}のように、スペルミスにより\texttt{"Hello"}ではなく\texttt{"Hell"}という文字列を持つ
\texttt{String}オブジェクトを削除してしまった場合、
23行目で配列から\texttt{"Hello"}という文字列を保持する\texttt{String}オブジェクトが削除されず、
24行目で\texttt{h}のメモリ領域が解放されないため
メモリリークが起きてしまう。
\begin{figure}[htp]
\begin{lstlisting}[caption=ユーザのミスによるメモリリーク, 
  label=rc_miss_problem, captionpos=b]
use std::rc::Rc;
use rand::Rng;
fn remove_string(vec: &mut Vec<Rc<String>>, s: String) {
    vec.retain(|x| **x != s);
}
  
fn main() {
    let mut vec: Vec<Rc<String>> = Vec::new();
    {
      let h = Rc::new(String::from("Hello"));
      let w = Rc::new(String::from("World"));
      let mut rng = rand::thread_rng();
      for _ in 0..10 {
          if rng.gen() {
              let h_clone = h.clone();
              vec.push(h_clone);
          }
          else {
              let w_clone = w.clone();
              vec.push(w_clone);
          }
      }
      remove_string(&mut vec, String::from(@*"Hell"*@)); // スペルミス
    } // hもwもどちらも中身のオブジェクトは解放されない
    // その他の処理
}
\end{lstlisting}
\end{figure}
% 参照カウントだと、ミスが起こりやすい
% 静的検査が望ましいがない、せめて動的検査があってほしい
% into_inner入れればできるが、忘れたらいけない

% ここは実装の部分に移す
また、\texttt{Rc}オブジェクトの中身のオブジェクトを取り出す
\texttt{into\_inner}関数を用いることで、
図\ref{rc_problem}のプログラムは図\ref{rc_random}のように、
メモリリークが起きないように表すことができる。
24行目の部分で、複製が消滅した後、25行目で\texttt{into\_inner}関数
により中身のオブジェクトを取り出している。
複製が消滅していなければ、\texttt{into\_inner}を用いる際にエラーが起きるため、意図していないメモリリークを検出することができる。
しかし、\texttt{into\_inner}関数により中身のオブジェクトを取り出すことを忘れてしまうと、メモリリークが発生してしまう。
\begin{figure}[htp]
\begin{lstlisting}[caption=into\_inner関数を用いた例, 
  label=rc_random, captionpos=b]
use std::rc::Rc;
use rand::Rng;
fn remove_string(vec: &mut Vec<Rc<String>>, s: String) {
    vec.retain(|x| **x != s);
}

fn main() {
  let mut vec: Vec<Rc<String>> = Vec::new();
  {
    let h = String::from("Hello");
    let w = String::from("World");
    let mut rng = rand::thread_rng();
    let h_immut = Rc::new(h);
    let w_immut = Rc::new(w);
    for _ in 0..10 {
      if rng.gen() {
        let h_clone = h_immut.clone();
        vec.push(h_clone);
      }
      else {
        let w_clone = w_immut.clone();
        vec.push(w_clone);
      }
    }
    remove_string(&mut vec, String::from("Hello")); // 参照が消滅
    @*let mut h_mut = Rc::into_inner(h_immut).unwrap();*@ // into_inner
  }
  // その他の処理
}
\end{lstlisting}
\end{figure}

このように、\texttt{Rc}オブジェクトを用いた場合、
ユーザによるミスが起こりやすく、結果としてメモリリークが起きる場合がある。
現在、Rustでこのようなメモリリークが起きないことは静的に検査する手段が存在しないだけでなく、
動的に検査する手段も存在していない。
また、Rustはメモリリークを防止できるというメリットがあるが、
\texttt{Rc}オブジェクトを用いるとそのメリットは損なわれてしまう。

\chapter{本研究の提案}
本研究では、Rustの静的なownershipによるメモリ管理と、
動的な参照カウントの柔軟性を組み合わせた新たな参照オブジェクトを提案する。
この参照オブジェクトは、fractional ownershipの考え方を取り入れている。

\section{提案するインターフェース}
% 0120 check
\texttt{Rc}オブジェクトを用いた際に、ユーザのミスによるメモリリークが検出できない
という問題を解決するために、
図\ref{random_example}のように用いることができる参照オブジェクトを提案する。
このプログラムは、中身に\texttt{"Hello"}という\texttt{String}オブジェクトを持つ
mutableな参照オブジェクト\texttt{h}と、中身に\texttt{"World"}という\texttt{String}オブジェクトを持つ
mutableな参照オブジェクト\texttt{w}をどちらもimmutableな参照オブジェクトに変換した後、
複製をランダムに10回配列に入れ、
その配列から\texttt{"Hello"}という\texttt{String}オブジェクトを持つmutableな参照オブジェクトを
削除するプログラムである。
7、8行目で生成したmutableな参照オブジェクトを、10、11行目でimmutableな参照オブジェクトに変換し、
14、19行目でimmutableな参照オブジェクトを複製している。
\texttt{remove\_string}が終了した後、26行目、32行目でmutableな参照オブジェクトに戻している。
\begin{figure}[htp]
\begin{lstlisting}[caption=新たな参照オブジェクトを用いた柔軟なメモリ管理の例, 
  label=random_example, captionpos=b]
use fRef_implementation::fractional_ref::FRefImmut;
use fRef_implementation::fractional_ref::FRefMut;
use rand::Rng;
fn remove_string(vec: &mut Vec<FRefImmut<String>>, s: String) {
    vec.retain(|x| **x != s);
}

fn main() {
  let mut vec: Vec<FRefImmut<String>> = Vec::new();
  let mut h = FRefMut::new(String::from("Hello")); // Mutableな参照
  let mut w = FRefMut::new(String::from("World")); // Mutableな参照
  let mut rng = rand::thread_rng();
  let h_immut = h.to_immut(); // Mutableからimmutableへ変換
  let w_immut = w.to_immut(); // Mutableからimmutableへ変換
  for _ in 0..10 {
    if rng.gen() {
      let h_immut1 = h_immut.clone_immut(); // 参照の複製
      println!("{}", h_immut1);
      vec.push(h_immut1);
    }
    else {
      let w_immut1 = w_immut.clone_immut(); // 参照の複製
      println!("{}", w_immut1);
      vec.push(w_immut1);
    }
  }
  remove_string(&mut vec, String::from("Hello"));
  let mut h_mut = h_immut.back_to_mut(); // Immutableからmutableへ
  h_mut.push_str("hoge");
  println!("{}", h_mut);
  drop(h_mut);
  drop(vec);
  let mut w_mut = w_immut.back_to_mut(); // Immutableからmutableへ
}
\end{lstlisting}
\end{figure}

この参照オブジェクトは、読み書き可能であり、かつ解放が可能な\texttt{FRefMut}オブジェクトと
読み取りのみ可能である\texttt{FRefImmut}オブジェクトの2つがあり、静的に区別されている。
\texttt{FRefMut}オブジェクトはRustの通常のownershipに従い、同時に一つしか存在しない。
\texttt{FRefMut}オブジェクトは、7、8行目のように\texttt{new}関数により生成できる。
10、11行目のように\texttt{to\_immut}関数を用いると、\texttt{FRefMut}オブジェクトを\texttt{FRefImmut}オブジェクトに変換でき、
その際Rustの静的なownershipのmoveにより\texttt{FRefMut}オブジェクトは消滅する。
14、19行目のように、
\texttt{FRefImmut}オブジェクトは、Rustの通常のownershipに違反して\texttt{clone\_immut}関数により複製が可能であり、
複製の数を動的にカウントする。
26、32行目のように、
参照カウントが1つであるときのみ、
\texttt{back\_to\_immut}関数を用いて\texttt{FRefImmut}オブジェクトから\texttt{FRefMut}オブジェクトへ変換でき、
その際moveにより\texttt{FRefImmut}オブジェクトは消滅する。
\texttt{FRefImmut}オブジェクトから\texttt{FRefMut}オブジェクトに戻し忘れると、実行時エラーが発生するため、
オブジェクトの解放し忘れが検出できる。

また、図\ref{error_example2}は、図\ref{random_example}とほぼ同じプログラムであるが、
参照カウントが1でないにも関わらず\texttt{FRefImmut}オブジェクトを\texttt{FRefMut}オブジェクトに戻そうとしているプログラムである。
この場合、mutableな参照に戻すことができずに図\ref{error_result2}のような実行時エラーが発生するため、
\texttt{FRefMut}オブジェクトでないと消滅させることはできないようユーザに強制することができている。

図\ref{error_example1}は、図\ref{random_example}とほぼ同じプログラムであるが、
\texttt{FRefImmut}オブジェクトを\texttt{FRefMut}オブジェクトに戻していないプログラムである。
この場合、\texttt{FRefMut}オブジェクトに戻さないまま参照が消滅してしまっているため図\ref{error_result1}のように実行時エラーが発生し、
オブジェクトの解放し忘れを検出することができる。

\begin{figure}[htp]
\begin{lstlisting}[caption=オブジェクトの解放し忘れを検出する例, 
  label=error_example1, captionpos=b]
use fRef_implementation::fractional_ref::FRefImmut;
use fRef_implementation::fractional_ref::FRefMut;
use rand::Rng;
fn remove_string(vec: &mut Vec<FRefImmut<String>>, s: String) {
    vec.retain(|x| **x != s);
}

fn main() {
  let mut vec: Vec<FRefImmut<String>> = Vec::new();
  let h = FRefMut::new(String::from("Hello")); // FRefMutの生成
  let w = FRefMut::new(String::from("World")); // FRefMutの生成
  let mut rng = rand::thread_rng();
  let h_immut = h.to_immut(); // FRefMutからFRefImmutへ変換
  let w_immut = w.to_immut(); // FRefMutからFRefImmutへ変換
  for _ in 0..10 {
    if rng.gen() {
      let h_immut1 = h_immut.clone_immut(); // FRefImmutの複製
      println!("{}", h_immut1);
      vec.push(h_immut1);
    }
    else {
      let w_immut1 = w_immut.clone_immut(); // FRefImmutの複製
      println!("{}", w_immut1);
      vec.push(w_immut1);
    }
  }
  remove_string(&mut vec, String::from("Hello"));
  // FRefMutに戻していない
} // error!
\end{lstlisting}
\end{figure}

\begin{figure}[p]
\begin{lstlisting}[caption=オブジェクトの解放し忘れのエラー, 
  label=error_result1, captionpos=b]
$ cargo run
   Compiling dynamic_error_example2 v0.1.0 
    (/home/baba/experiment-report/202401/0105/dynamic_error_example2)
    Finished dev [unoptimized + debuginfo] target(s) in 0.94s
     Running `target/debug/dynamic_error_example2`
Hello
World
Hello
World
Hello
Hello
Hello
World
World
World
thread 'main' panicked at /home/baba/experiment-report/202401/0105
    /dynamic_error_example2/src/lib.rs:154:17:
memory leak
note: run with `RUST_BACKTRACE=1` 
    environment variable to display a backtrace
\end{lstlisting}
\end{figure}

\begin{figure}[p]
\begin{lstlisting}[caption=Mutableな参照へ戻せないことを検出する例, 
  label=error_example2, captionpos=b]
use fRef_implementation::fractional_ref::FRefImmut;
use fRef_implementation::fractional_ref::FRefMut;
use rand::Rng;
fn remove_string(vec: &mut Vec<FRefImmut<String>>, s: String) {
    vec.retain(|x| **x != s);
}

fn main() {
    let mut vec: Vec<FRefImmut<String>> = Vec::new();
    let h = FRefMut::new(String::from("Hello"));
    let w = FRefMut::new(String::from("World"));
    let mut rng = rand::thread_rng();
    let h_immut = h.to_immut();
    let w_immut = w.to_immut();
    for _ in 0..10 {
        if rng.gen() {
            let h_immut1 = h_immut.clone_immut();
            println!("{}", h_immut1);
            vec.push(h_immut1);
        }
        else {
            let w_immut1 = w_immut.clone_immut();
            println!("{}", w_immut1);
            vec.push(w_immut1);
        }
    }

    remove_string(&mut vec, String::from("Hello"));
    let mut w_mut = w_immut.back_to_mut(); // error!
    w_mut.push_str("hhhhh");
    println!("{}", w_mut);
}
\end{lstlisting}
\end{figure}
\begin{figure}[p]
\begin{lstlisting}[caption=FRefMutに戻し忘れた際のエラー, 
  label=error_result2, captionpos=b]
$ cargo run
   Compiling dynamic_error_example v0.1.0
    (/home/baba/experiment-report/202401/0105/dynamic_error_example)
    Finished dev [unoptimized + debuginfo] target(s) in 0.98s
     Running `target/debug/dynamic_error_example`
Hello
Hello
Hello
World
Hello
Hello
World
World
World
Hello
thread 'main' panicked at /home/baba/experiment-report/202401/0105
    /dynamic_error_example/src/lib.rs:72:17:
cannot back to mut
note: run with `RUST_BACKTRACE=1` 
    environment variable to display a backtrace
\end{lstlisting}
\end{figure}

図\ref{error_example3}は、図\ref{random_example}とほぼ同じプログラムであるが、
文字列を削除する際にスペルミスをしてしまっているプログラムである。
この場合、\texttt{FRefMut}に戻すことができずに図\ref{error_result3}のように実行時エラーが発生するため、
メモリリークを防ぐことができている。
\begin{figure}[p]
\begin{lstlisting}[caption=ユーザのミスを検出する例, 
  label=error_example3, captionpos=b]
use fRef_implementation::fractional_ref::FRefImmut;
use fRef_implementation::fractional_ref::FRefMut;
use rand::Rng;
fn remove_string(vec: &mut Vec<FRefImmut<String>>, s: String) {
    vec.retain(|x| **x != s);
}

fn main() {
    let mut vec: Vec<FRefImmut<String>> = Vec::new();
    let h = FRefMut::new(String::from("Hello"));
    let w = FRefMut::new(String::from("World"));
    let mut rng = rand::thread_rng();
    let h_immut = h.to_immut();
    let w_immut = w.to_immut();
    for _ in 0..10 {
        if rng.gen() {
            let h_immut1 = h_immut.clone_immut();
            println!("{}", h_immut1);
            vec.push(h_immut1);
        }
        else {
            let w_immut1 = w_immut.clone_immut();
            println!("{}", w_immut1);
            vec.push(w_immut1);
        }
    }

    remove_string(&mut vec, String::from("Hell")); // スペルミス
    let mut h_mut = h_immut.back_to_mut(); // error!
    h_mut.push_str("hoge");
    println!("{}", h_mut);
    drop(h_mut);

    // その他の処理

    drop(vec);
    let _w_mut = w_immut.back_to_mut();
}
\end{lstlisting}
\end{figure}
\begin{figure}[p]
\begin{lstlisting}[caption=ユーザのミスによるエラー, 
  label=error_result3, captionpos=b]
$ cargo run
   Compiling random v0.1.0 
    (/home/baba/experiment-report/202401/0105/random)
warning: unused variable: `w_new`
  --> src/main.rs:39:9
   |
39 |     let w_new = w_immut.back_to_mut();
   |         ^^^^^ help: if this is intentional,
    prefix it with an underscore: `_w_new`
   |
   = note: `#[warn(unused_variables)]` on by default

warning: `random` (bin "random") generated 1 warning 
    (run `cargo fix --bin "random"` to apply 1 suggestion)
    Finished dev [unoptimized + debuginfo] target(s) in 0.67s
     Running `target/debug/random`
Hello
Hello
Hello
World
World
World
Hello
Hello
Hello
Hello
thread 'main' panicked at /home/baba/experiment-report/202401/0105
    /random/src/lib.rs:72:17:
cannot back to mut
note: run with `RUST_BACKTRACE=1` 
    environment variable to display a backtrace
\end{lstlisting}
\end{figure}

\afterpage{\clearpage}
\newpage
\section{実装}
% 0120 check
\texttt{FRefMut}オブジェクト、\texttt{FRefImmut}オブジェクトは図\ref{struct_implementation}のように実装した。
\texttt{FRefMut}オブジェクトは、オブジェクトを保持するコンテナとして実装した。
\texttt{FRefImmut}オブジェクトは、
Rustのownershipを無視する特別なポインタであるNonNullオブジェクトと、
PhantomDataを保持するオブジェクトとして実装した。
NonNullオブジェクトにより、参照の複製が可能になる。
また、PhantomDataにより、\texttt{FRefImmut}がT型の値を保持することを
コンパイラに示している。
\texttt{FRefImmut}内のポインタは\texttt{FRefInner}オブジェクトへのポインタであり、
このオブジェクトは参照カウントと中身のオブジェクトを保持する。
\begin{figure}[htp]
\begin{lstlisting}[caption=新たな参照オブジェクトの実装, label=struct_implementation, captionpos=b]
pub struct FRefMut<T: ?Sized> {
    data: T,
}
pub struct FRefImmut<T: ?Sized> {
    ptr: NonNull<FRefInner<T>>,
    phantom: PhantomData<FRefInner<T>>
}
struct FRefInner<T: ?Sized> {
    ref_count: atomic::AtomicUsize,
    data: T
}
\end{lstlisting}
\end{figure}

オブジェクトを引数として受け取り、そのオブジェクトを中身として持つ\texttt{FRefMut}オブジェクトを生成する\texttt{new}関数と、\texttt{FRefMut}オブジェクトを引数として受け取り、
\texttt{FRefImmut}オブジェクトに変換して返す\texttt{to\_immut}関数は
図\ref{new_to_implementation}ように実装した。
4行目で\texttt{to\_immut}関数の引数は、borrowingでなくmoveが起きるようにすることで、
変換した後引数の\texttt{FRefMut}が消滅するように実装した。

\texttt{FRefImmut}オブジェクトを引数として受け取り、
複製した参照を返す\texttt{clone\_immut}関数は図\ref{clone_implementation}ように実装した。
\texttt{clone\_immut}関数の引数をborrowingにすることで、複製元の参照が消滅せずに利用できるように実装した。
また、2行目の\texttt{fetch\_add}関数によって\texttt{FRefImmut}の内部の参照カウントが1増加するように実装した。

\texttt{FRefImmut}オブジェクトを引数として受け取り、
\texttt{FRefMut}オブジェクトに変換する\texttt{back\_to\_mut}関数は
図\ref{back_implementation}のように実装した。
\texttt{back\_to\_mut}関数の引数をborowwingでなくmoveするようにし、
変換した後引数の\texttt{FRefImmut}が消滅するように実装した。
また、2行目で\texttt{FRefImmut}の参照カウントが1でなければ実行時エラーが起きるようにし、
\texttt{FRefImmut}のままでは消滅できないように実装した。

\texttt{FRefImmut}オブジェクトを引数として受け取り、
その参照を消滅させる\texttt{drop}関数は図\ref{drop_implementation}のように実装した。
参照カウントが2以上である場合は、参照カウントを1減少させ、
参照カウントが1である場合は実行時エラーが起きるようにすることで、
\texttt{FRefImmut}を\texttt{FRefMut}に戻さずに消滅できなように実装した。
これにより、オブジェクトの解放し忘れを検出することが可能である。

\begin{figure}[htp]
\begin{lstlisting}[caption=関数newと関数to\_immutの実装, label=new_to_implementation, captionpos=b]
pub fn new(data: T) -> FRefMut<T> {
    Self { data: data }
}
pub fn to_immut(self: FRefMut<T>) -> FRefImmut<T>{ // Moveにより消滅
    let mut this = ManuallyDrop::new(self);
    let inner =
        unsafe{ptr::read(&mut this.data)};
    let x = Box::new(FRefInner {
        ref_count: AtomicUsize::new(1),
        data: inner,
    });
    FRefImmut {ptr: Box::leak(x).into(), phantom:PhantomData}
}
\end{lstlisting}
\end{figure}

\begin{figure}[htp]
\begin{lstlisting}[caption=関数clone\_immutの実装, label=clone_implementation, captionpos=b]
pub fn clone_immut(&self: &FRefImmut<T>) -> FRefImmut<T>
{
    let old_size = self.inner().ref_count.fetch_add(1, Release);
    if old_size > MAX_REFCOUNT {
        abort();
    }
    FRefImmut { ptr: self.ptr, phantom: PhantomData }
}
\end{lstlisting}
\end{figure}

\begin{figure}[htp]
\begin{lstlisting}[caption=関数back\_to\_mutの実装, label=back_implementation, captionpos=b]
pub fn back_to_mut(self: FRefImmut<T>) -> FRefMut<T> { // Move
    if self.inner().ref_count.load(Acquire) != 1{
        panic!("cannot back to mut");
    }
    let mut this = ManuallyDrop::new(self);
    let inner =
        unsafe { ptr::read(&mut (*this.ptr.as_ptr()).data)};
    unsafe {
        dealloc(this.ptr.cast().as_mut(), 
                Layout::for_value_raw(this.ptr.as_ptr()))
    }
    FRefMut { data: inner }
}
\end{lstlisting}
\end{figure}

\begin{figure}[htp]
\begin{lstlisting}[caption=関数dropの実装, label=drop_implementation, captionpos=b]
impl<T: ?Sized> Drop for FRefImmut<T> {
    fn drop(&mut self) {
        let count = self.inner().ref_count.fetch_sub(1, Relaxed);
        atomic::fence(Acquire);
        if count > 1{
            return
        }
        else if count == 1{
            unsafe { ptr::drop_in_place(self.ptr.as_ptr()) };
            if !panicking() {
                panic!("memory leak");
            }
        }
        else {
            abort();
        }
    }
}
\end{lstlisting}
\end{figure}

なお、実装からわかるように提案する参照オブジェクトは\texttt{Rc}オブジェクトの利用方法を強制しているオブジェクトであることから、\texttt{Rc}オブジェクトと\texttt{Rc}オブジェクトの中身を取り出す関数\texttt{into\_inner}を用いて実装することが可能のように思える。
その際、図\ref{into_inner}のようにimmutableな参照オブジェクトのフィールドが\texttt{Rc}オブジェクトになる。
Immutableな参照からmutableな参照に戻す際には、フィールドに対して\texttt{into\_inner}を呼び出す必要があり、
\texttt{into\_inner}の引数は、borrowingでなくmoveが起きることから、オブジェクトのフィールドはmoveする。
そのため、本来オブジェクトが解放された後にそのフィールドが解放されるはずだが、オブジェクトの解放より前にフィールドが解放されることになる。
提案する参照オブジェクトでは、
immutableな参照を消滅させる\texttt{drop}関数内で、
オブジェクトの解放し忘れを検出するためにmutableな参照に戻し忘れていないか確認する必要があり、
\texttt{drop}関数によってオブジェクトの解放の際にフィールドを用いた処理が行われる可能性がある。
この場合、すでに解放されているフィールドを用いる処理が行われる可能性があるため、Rustでは静的なエラーが発生する。
なお、オブジェクトとフィールドの解放を同時に行うように書くことはできない。
従って、\texttt{Rc}オブジェクトを用いて提案する参照オブジェクトを実装する場合、\texttt{Rc}オブジェクト以外にunsafeな機能を用いる必要がある。

\begin{figure}[htp]
\begin{lstlisting}[caption=Rcオブジェクトと関数into\_innerによる実装の試み, label=into_inner, captionpos=b]
pub struct FRefMut<T: ?Sized> {
    data: T,
}
pub struct FRefImmut<T: ?Sized> {
    rc: Rc<T>
}
\end{lstlisting}
\end{figure}

\chapter{並列処理への応用}
% 0120 check
静的なownershipで柔軟性に欠ける例として、並列処理の例があり、本研究の新たな参照オブジェクトを応用できる。

\section{Rustの並列処理}
まず、Rustでどのように並列処理を行うことができるかを述べる。
\subsection{通常のスレッド}
Rustの通常のスレッド生成では、ownerは高々一つのスレッドにmoveし、
複数のスレッドにmoveすることはできない。図\ref{thread_example}は、
Rustで通常のスレッドを生成する例であり、3行目にある\texttt{move}キーワードによりmoveが起きる。
\begin{figure}[htp]
\begin{lstlisting}[caption=Rustでの通常のスレッド生成の例, 
  label=thread_example, captionpos=b]
use std::thread;
fn main() {
    let s = String::from("Hello");
    thread::spawn(move || { // sがスレッドにmove
        println!("{}", s);
    });
}
\end{lstlisting}
\end{figure}

Rustでは、スレッドのライフタイムは無限とみなされており、スレッドが合流した際にownershipが返却されないため、
共有オブジェクトに対するborrowingは不可能である。
図\ref{thread_borrow_example}は、
Rustで通常のスレッドで共有オブジェクトに対して
borrowingをしようとしている例である。
4、7行目でborrowingを行っているが、
9、10行目でownershipが返却されず、コンパイルエラーとなる。
\begin{figure}[htp]
\begin{lstlisting}[caption=通常のスレッドでborrowingができない例, 
  label=thread_borrow_example, captionpos=b]
use std::thread;
fn main() {
    let s = String::from("Hello");
    let t1 = thread::spawn(|| {
        println!("{}", &s); // コンパイルエラー
    });
    let t2 = thread::spawn(|| {
        println!("{}", &s); // コンパイルエラー
    });
    t1.join(); // Ownershipが返却されない
    t2.join(); // Ownershipが返却されない
}
\end{lstlisting}
\end{figure}

\subsection{scoped thread}
共有オブジェクトに対するborrowingができないことは不便であるため、
Rustにはスレッドをまたいだborrowingを可能にするscoped threadが存在する。
これは、スレッドに静的スコープを付与したものであり、静的スコープに従ってスレッドの合流が起きるため、
共有オブジェクトに対するborrowingが可能となる。
図\ref{scoped_thread_example}は、scoped threadを利用した例である。
スレッドに付与されたスコープは、10行目で終了しスレッドの合流が起こる。
\begin{figure}[htp]
\begin{lstlisting}[caption=Scoped threadの例, label=scoped_thread_example, captionpos=b]
use std::thread;
fn main () {
    let mut s = String::from("Hello");
    thread::scope(|sc| { // scがスコープ
        sc.spawn(|| {
            println!("{}", &s);
        });
        sc.spawn(|| {
            println!("{}", &s);
        });
    }); // スコープが終了し、スレッドが合流
    s.push_str("World");
}
\end{lstlisting}
\end{figure}

\section{並列処理の際の問題点}
図\ref{scoped_thread_problem}はscoped threadで表すことができない。
関数\texttt{my\_spawn}は、引数として\texttt{String}オブジェクトを受け取り、スレッドを生成し、
スレッドのハンドラを返す関数である。
このプログラムは、関数\texttt{my\_spawn}により\texttt{main}関数外で生成したスレッドが
\texttt{main}関数内で合流するというプログラムである。
このプログラムは、スレッドの合流が静的スコープに従っていないため、scoped threadで表すことができない。
\begin{figure}[htp]
\begin{lstlisting}[caption=scoped threadで表せない例, 
  label=scoped_thread_problem, captionpos=b]
use std::thread;
use std::thread::JoinHandle;
fn my_spawn (a: &String) -> JoinHandle<()>{
    let t = thread::spawn(|| {
        println!("{}", &a);
    });
    return t
}
fn main() {
    let mut a = String::from("Hello");
    let t1 = my_spawn(&a);
    let t2 = my_spawn(&a);
    t1.join();
    t2.join();
    a.push_str("World");
}
\end{lstlisting}
\end{figure}

\section{我々の提案の応用}
スレッドの合流が静的スコープに従わない並列処理は、
本研究の新たな参照オブジェクトにより図\ref{scoped_example}のように表すことができる。
関数\texttt{my\_spawn}は、引数として\texttt{String}オブジェクトを保持する\texttt{FRefImmut}な参照を受け取り、
スレッドを生成し、スレッドのハンドラを返す関数である。
このプログラムは、関数\texttt{my\_spawn}により\texttt{main}関数外で生成したスレッドが
\texttt{main}関数内で合流するというプログラムである。
16行目で\texttt{FRefMut}を生成し、17、18行目で\texttt{FRefImmut}に変換した後、複製している。
その後、関数\texttt{my\_spawn}によりスレッドを生成し、
21、22行目で\texttt{main}関数内でスレッドが合流している。
その後関数\texttt{drop}によりownershipの集約が起き、
関数\texttt{back\_to\_mut}により\texttt{FRefMut}に戻している。
\begin{figure}[htp]
\begin{lstlisting}[caption=新たな参照オブジェクトを用いた並列処理の例, 
  label=scoped_example, captionpos=b]
#![feature(thread_spawn_unchecked)]
use fRef_implementation::fractional_ref::FRefImmut;
use fRef_implementation::fractional_ref::FRefMut;
use std::thread;
use std::thread::JoinHandle;
fn my_spawn (a: &FRefImmut<String>) -> JoinHandle<()>{
  let builder = thread::Builder::new();
  unsafe {
    let t = builder.spawn_unchecked(||{
      println!("{}", a);
    }).unwrap();
    return t
  }
}
fn main() {
  let mut s = FRefMut::new(String::from("Hello")); // FRefMutの生成
  let s1 = s.to_immut(); // FRefMutからFRefImmutへ変換
  let s2 = s1.clone_immut(); // FRefImmutの複製
  let t1 = my_spawn(&s1);
  let t2 = my_spawn(&s2); 
  t1.join();
  t2.join();
  drop(s2);
  let mut s3 = s1.back_to_mut(); // FRefImmutからFRefMutへ変換
  s3.push_str("World");
}
\end{lstlisting}
\end{figure}

\subsection{Readers-Writer Lock}
Rustには、Readers-Writer Lock\cite{DBLP:journals/cacm/CouroisHP71}がある。
これは、複数の読み取りは許すが書き込みのアクセスの際は排他的な場合のみ許すロックである。
図\ref{rwlock}は、図\ref{scoped_example}をReaders-Writer Lockを用いて表したプログラムである。
しかし、Readers-Writer Lockを用いた際には、
16行目のように本来不要なロックが発生するおそれがある。
本研究の提案を応用すると、
ロックではなく\texttt{FRefMut}と\texttt{FRefImmut}の
区別により競合状態を防止できる。
\begin{figure}[htp]
\begin{lstlisting}[caption=Readers-Writer Lockを用いた並列処理の例, 
  label=rwlock, captionpos=b]
#![feature(thread_spawn_unchecked)]
use std::thread;
use std::thread::JoinHandle;
use std::sync::RwLock;
fn my_spawn (a: &String) -> JoinHandle<()>{
    let builder = thread::Builder::new();
    unsafe {
        let t = builder.spawn_unchecked(||{
            println!("{}", a);
        }).unwrap();
        return t
    }
}
fn main() {
    let mut s = RwLock::new(String::from("Hello"));
    let s1 = s.read().unwrap();
    let s2 = s.read().unwrap();
    let t1 = my_spawn(&s1);
    let t2 = my_spawn(&s2); 
    t1.join();
    t2.join();
    drop(s1);
    drop(s2);
    let mut s3 = s.write().unwrap(); // 不要なロック
    s3.push_str("World");
}
\end{lstlisting}
\end{figure}

\chapter{関連研究}
% 0120 check
本研究で取り入れた、fractional ownership\cite{DBLP:conf/sas/Boyland03}は、
有理数計画法を用いて静的に検査されている。
また、RustでなくC言語の特定のプログラムでメモリ関連のエラーがないことが
fractional ownershipにより検証されている\cite{10.1007/978-3-642-10672-9_11}。
また、fractional ownershipとRustの静的なownershipを組み合わせた新たな所有権型が提案されており、
提案した型システムにより命令型プログラムが検証されている\cite{10.1007/978-3-031-50521-8_11}。
これらはすべて動的検査でなく静的検査であり、保守的であるものの本研究のような動的検査と比べると
柔軟性に欠ける検査となっている。

また、RustBelt\cite{DBLP:journals/pacmpl/0002JKD18}によりunsafeな機能を用いた際でもメモリ安全であるか証明できる。
本研究の新たな参照オブジェクトがメモリ安全であるかをRustBeltを用いて証明できるか検討した。
\texttt{Rc}オブジェクトは実装にunsafeな機能が用いられているが、
RustBeltによってメモリ安全であることが証明されている。
本研究の参照オブジェクトは、\texttt{Rc}オブジェクトの利用方法を強制しているオブジェクトであるため、
\texttt{Rc}オブジェクトを用いて本研究の参照オブジェクトを実装できれば、メモリ安全であることが証明できると考えた。
しかし、unsafeな機能を用いなければ実装することができなかったため、
本研究の参照オブジェクトがメモリ安全であるかは、RustBeltを用いて新たに証明する必要があった。

\chapter{結論と今後の課題}
Fractional ownershipの考え方を取り入れ、Rustの静的なownershipによるメモリ管理と、
動的な参照カウントの柔軟性を組み合わせた新たな参照オブジェクトを提案した。
この参照オブジェクトにより、オブジェクトの解放し忘れを実行時に検出し、メモリリークを防ぐことができる。
今後の課題は、新たな参照オブジェクトがメモリ安全であることを証明することが挙げられる。
また、新たな参照オブジェクトをより具体的な例で利用し、どれほどメモリリークを防ぐことができたかを実験することも
挙げられる。

\backmatter% ここから後付
\chapter{謝辞}
本論文の執筆にあたりご指導くださった住井英二郎教授に感謝申し上げます。
また、論文の執筆にあたりご指摘くださった松田一孝准教授、
ゼミや発表でご指摘くださったOleg Kiselyov助教、
住井・松田研究室の皆様に感謝申し上げます。
% dblpで確認してみる
% % %% 参考文献: bibtex
\bibliographystyle{jplain}
\bibliography{refs}

\end{document}
